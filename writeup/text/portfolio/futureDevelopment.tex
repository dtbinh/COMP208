\section{Interface Framework}
Currently the interface framework is the Google Web Toolkit (GWT).  This
currently allows the Turtlenet client's interface to run in a web browser of the
user's choosing.  This choice helped the group and the project initially by 
allowing the developers to not consciously worry about any particular problems
in the running of the project in terms of the front end - our mentality was that
everyone had a browser on their computers so it was one less requirement to add
to the minimum requirements section of the user manual.  In hindsight however,
what this decision has done is open the project to a flood of potential bug
reports; some from inexperienced users hoping for a fix on Internet Explorer
whilst others filling in reports about how their personally compiled browser is
not rendering a button, for example.  With Java, the code is run by a Java
Virtual Machine (JVM) so there is only one platform to worry about whereas
through the use of a browser being a container for the user interface, each with
different layout engines and capabilities, the project can expect to receive bug
reports from at least four different front ends.

With the potential workload heavily increased just by a decision made by the
user, which we assume holds little to no knowledge and just wishes the project
work on their computer, this may be a problem for the future developers of
Turtlenet.  As a possible solution, a different framework could be used which
provides an executable that is tailored to the operating system, as opposed to
a web browser.  Running in a native window as opposed to a browser means that
there is only a couple of different ways the front end can appear, variations
mainly appearing due to a change of operating system - although a different
desktop environment (DE) would also affect the final look of Turtlenet.  This
improves the usability of the project as the final outcome would be an
executable file, which runs in the same or similar manner to other programs on
their computer, improves consistency as GWT creates JavaScript as a core 
component in the running of the front-end, which may be blocked by the browser
as well as creating specific statements which are interpreted or ignored
dependent on the browser currently in use.  Finally there is debugging, which
is improved because there are fewer main variables for the testers to cater to,
such as versions of browsers and their plug-in files which may hamper use of the
front end.  Through consolidation into a native window, testing and bug catching
can become more efficient as there are less programs to load, and less time
required.

\section{Interface and House Style}
The current interface for the client has been made in such a way that it
provides all of the functionality of the project in fairly easily identifiable
sections.  What it does not do though is look polished enough to be on an
average user's computer yet.  This is most likely due to time constraints
stressing for functionality as opposed to aesthetics.  Another potential problem
is the house style of the front-end.  Green has symbolic meaning and was not
chosen simply due to the name of the project - Turtles more often than not have
darker colours such as brown or grey and not green.  Green is often used in
healthcare as a sign that something is either safe or good for you (the green
health 'plus' being an example), which is what Turtlenet aims to be for your
communicative efforts.  On a per-user basis however, colour is simply a way of
making the front end become more pleasant.

This leads to the problem of personal taste - some people don't like green.
Therefore to increase usability of the project as well as the total amount of
users, themes could be a future development.  Allowing the user to change the
look of the front end can make a difference to the amount of people using the
project.  More users may appear if the project synchronises well with the rest
of their system.

\section{Languages Used}
The project used Java for the back end of the system, SQLite for the Database
and Java converted to JavaScript for the front end.  Java was chosen for the
interoperability of the language - being able to run on whatever has a Java
Virtual Machine (JVM), which are available for most operating systems.  Most
users have the Java Runtime Environment (JRE) installed, which includes JVM so
Java was a good choice for the project.

SQLite is a notably lightweight Database Management System (DBMS) at the expense
of some features that are used in a more complete SQL solution, none of which
were needed for the project.  SQL notably requires you to define data type as
well as the length of the variable as well - while SQLite is more lenient in
this regard, removing this constraint completely would make a system more
usable.  An example of a more user friendly database would be one that uses 
MongoDB but that is not as popular as an SQL derived DBMS, so this is the
reason SQLite was chosen.

Google Web Toolkit (GWT) allowed one of our developers the capability of writing
code in a similar manner to Java which when compiled creates the required
JavaScript and Ajax code.  On a technical level we believed this to be quite
clever, and was one of the reasons we chose GWT for the interface framework.
In hindsight it would have been better to choose a different framework which
would allow us to get a native executable after compilation.  This would mean
that the user does not need to open a terminal and enter any Java commands,
improving usability by not forcing the user to enter an environment that they
are not comfortable with.