\section{Interface Framework}
Currently the interface framework is the Google Web Toolkit (GWT).  This
currently allows the Turtlenet client's interface to run in a web browser of the
user's choosing.  This is because users are already familiar and at home in
their web browsers. This however comes with the downside that different browsers
may have different bugs. With a pure Java GUI, the code is run by a Java
Virtual Machine (JVM) so there is only one platform to worry about whereas
through the use of a browser being a container for the user interface, each with
different layout engines and capabilities, the project can expect to receive bug
reports from at least four different front ends. This is however significantly
mitigated by the fact that GWT compiles different versions of the javascript
source for each large browser, that takes their idiosyncrasies into account.

Given the current difficultly for the avarage user in launching turtlenet, it is
worthwhile to create a small executable that is tailored to each major operating
system and does nothing but start the client and open a web broser to the right
page.

If we were to move to a native GUI then while debugging would be easier, users
wouldn't have a consistant experiance or the comfort of their web browser,
something they already know how to use.

\section{Interface and House Style}
The current interface for the client has been made in such a way that it
provides all of the functionality of the project in fairly easily identifiable
sections.  What it does not do though is look polished enough to be on an
average user's computer yet.  This is most likely due to time constraints
stressing for functionality as opposed to aesthetics, however while not perfect
the current GUI does look far nicer than a native application would and isn't at
all hard on the eyes.

Green has symbolic meaning and was not chosen simply due to the name of the
project - Turtles more often than not have darker colours such as brown or grey
and not green.  Green is often used in healthcare as a sign that something is
either safe or good for you (the green health 'plus' being an example), which is
what Turtlenet aims to be for your communicative efforts.

We feel that the use of the colour green is calming, and the pervasive use of
the colour throughout the GUI leads to a consistant user experiance which builds
a brand identity. A promenant example of this concept is the fact that cadbury
have obtained a UK trademark on Pantone 2685C (purple).

\section{Languages Used}
The project used Java for the back end of the system, SQLite for the Database
and Java converted to JavaScript for the front end.  Java was chosen for the
interoperability of the language - being able to run on whatever has a Java
Virtual Machine (JVM), which are available for most operating systems.  Most
users have the Java Runtime Environment (JRE) installed, which includes JVM so
Java was a good choice for the project.

SQLite is a notably lightweight Database Management System (DBMS) at the expense
of some features that are used in a more complete SQL solution, none of which
were needed for the project.  SQL notably requires you to define data type as
well as the length of the variable as well - sqlite removing this constraint
allowed us to remove field limits without kludgy workarounds.

MongoDB is an attractive alternative, however the relative popularity of SQLite,
combined with the teams prior experiance with it, and greater choice of
libraries contributed to our ultimate decision to use SQLite.

Google Web Toolkit (GWT) allowed one of our developers the capability of writing
code in Java which when compiled creates the required AJAX code which makes RPC
calls to the java backend.  On a technical level we believed this to be quite
elegent, and was one of the reasons we chose GWT for the interface framework.
