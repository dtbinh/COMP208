\section{Network Architecture}
Turtlenet is a centralized service, whereby a large number of clients connect to
a single server which provides storage, and facilitates communication between
clients.

Due to the inherently limited network size (5-50K users per server depending on
percentage of active participants vs consumers and local internet speeds) we
recommend that servers serve a particular interest group or geographic locality.

Clients send messages to, and only to, these central servers. Due to the fact
that all messages (except CLAIM messages, see client-server/client-client
protocols for details) are encrypted the server does not maintain a database, it
cannot; rather clients each maintain their own local database, populated with
such information to which they have been granted access.

When a client wishes to send a message to a person, they encrypt the message with
the public key of the recipient\footnote{using RSA/AES, see protocol for
details} and upload it to the server. It is important to note that all network
connections are performed via Tor.

When a client wishes to view messages sent to them, they download all messages
posted to the server since they last downloaded all messages from it, and attempt
to decrypt them all with their private key; those messages the client
successfully decrypts (message decryption/integrity is verified via SHA256 hash)
were intended for it and parsed. During the parsing of a message the sender is
determined by seeing which known public key can verify the RSA signature.

Due to the nature of data storage in client-local databases, all events and data
within the system must be represented within these plaintext messages. This is
achieved by having multiple types of messages (see client-client protocol).

\section{System Architecture}
The system has a number of modules which interact with one another via strictly
defined interfaces. Each module has one function, and interacts as little as
possible with the rest of the system. The modules and their interactions are
shown below.
NB: a->b denotes that data passes from module a to module b, and a<->b similarly
denotes that data passes both from a to b and from b to a.

\begin{figure}[h]
\centering
\begin{tikzpicture} [node distance=.5cm, start chain=going below,]
    \node[punktchain, join] (serv)  {Turtlenet Server};
    \node[punktchain, join] (tor)   {Tor}
        edge[pil] (serv.south);
    \node[punktchain, join] (netc)  {NetworkConnection}
        edge[pil] (tor.south);
    \node[punktchain, join] (crypt) {Crypto}
        edge[pil] (netc.south);
    \node[punktchain, join] (parse) {Parser};
    \node[punktchain, join] (db)    {Database};
    \node[punktchain, join] (gserv) {Local HTTP Server}
            edge[pil,bend left=55] (netc.west)
            edge[pil,bend right=45] (db.east);
    \node[punktchain, join] (brows) {Web Browser}
            edge[pil] (gserv.south);
\end{tikzpicture}
\label{module_diagram}
\caption{Module Interaction}
\end{figure}

