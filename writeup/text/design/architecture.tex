\section{Network Architecture}
Turtlenet is a centralized service, whereby a large number of clients connect to
a single server which provides storage and facillitates communication between
clients.

Due to the inherantly limited network size (5-50K users per server depending on
percentage of active participants vs consumers and local internet speeds) we
reccomend that servers serve a particular interest group or geographic locality.
The client will allow inter-server communication completely transparently such
that while the users knows the networks they are part of, they are unaware of
the details of how communication takes place between them.

Clients send messages to, and only to, these central servers. Due to the fact
that all messages (except CLAIM messages, see client-server/client-client
protocols for details) are encrypted the server does not maintain a databse, it
cannot; rather clients each maintain their own local databased populated with
such information to which they have been granted access.

When a client wishes to send a message to a person they encrypt the message with
the public key of the recipient\footnote{using RSA/AES, see protocol for
details} and upload it to the server.

When a client wishes to view messages sent to them, they download all messages
posted to the server since they last downloaded all messages from it and attempt
to decrypt them all with their private key; those messages the client
successfully decrypts (message decryption/integrity is verified via SHA256 hash)
where intended for it and parsed. During the parsing of a message the sender is
determined by seeing which known public key can verify the RSA signature.

Due to the nature of data storage in client-local databases, all events and data
within the system must be represented within these plaintext messages. This is
achieved by having multiple types of messages (see client-client protocol).

\section{System Architecture}
The system has a number of modules which interact with one another via strictly
defined interfaces. Each module has one function, and interacts as little as
possible with the rest of the system. The modules are:
\todo{Update continually, this section is likely to fall into error at some
      point}
\begin{itemize}
    \item GUI Server (a simplified HTTP server)
    \item Crypto
    \item Database
    \item NetworkConnection
    \item Parser
    \item Tor client (seperate process, written in python)
    \item Server (on a different machine (central server))
\end{itemize}

The interactions are here shown:
\missingfigure{the interactions are not here shown :(\\
                 \textit{transcribe from notebook}}
