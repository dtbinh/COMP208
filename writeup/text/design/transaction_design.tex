The table below shows the transaction details of each function which will be 
found in the program. There are four types of transactions for databases which 
are insert, read, update and delete. 

Insertion is done when new data is added into a NULL attribute. Read on the 
other hand, is to view information from selected table(s) and its attribute(s).
Similar as insertion but update is conducted when data already exists in the 
particular attribute. This basically removes previous data and add a new one.
Lastly, delete, as it is self explanatory, deletes the whole tuple from the 
database. However this is usually avoided in database norms.

\begin{center}
    \begin{tabular}{ | p{3cm} | p{5cm} | p{3cm} |}
    \hline
    {\bf Function}        & {\bf Table(s) involved} & {\bf Transaction(s)} \\ \hline
    addClaim()      &                   &                \\ \hline 
    getClaims       &                   &                \\ \hline 
    getUsernames    & user              & Read           \\ \hline 
    addRevocation   & key\_revoke       & Insert         \\ \hline 
    getRevocations  & key\_revoke       & Read           \\ \hline 
    isRevoked()     & key\_revoke       & Insert         \\ \hline 
    addPData()      &                   &                \\ \hline 
    getPData()      &                   &                \\ \hline 
    createChat()    & private\_message  &  Insert        \\ \hline 
    getChat()       & private\_message, is\_in\_message             & Read           \\ \hline 
    addToChat()     & is\_in\_message   &   Insert       \\ \hline 
    addPost()       & wall\_post        &  Insert        \\ \hline 
    getPosts()      & wall\_post        &    Read        \\ \hline 
    \end{tabular}
\end{center}

\begin{center}
    \begin{tabular}{ | p{3cm} | p{5cm} | p{3cm} |}
    \hline
    {\bf Function}        & {\bf Table(s) involved} & {\bf Transaction(s)} \\ \hline
    addComment()    & has\_comment      &   Insert       \\ \hline 
    getComments()   & has\_comment      &  Read          \\ \hline 
    addLike()       & has\_like         &  Insert        \\ \hline 
    getLikes()      & has\_like         &  Read          \\ \hline 
    countLikes()    & has\_like         &  Read and count\\ \hline 
    addEvent()      & events            &    Insert      \\ \hline 
    getEvent()      & events            &  Read          \\ \hline 
    acceptEvent()   & events            &   Update       \\ \hline 
    declineEvent()  & events            &  Update        \\ \hline 
    addKey()        &                   &                \\ \hline 
    getKey()        &                   &                \\ \hline 
    getName()       & user              &  Read          \\ \hline 
    addFriend()     &                   &                \\ \hline 
    addCategory()   & category          &     Insert     \\ \hline  
    addToCategory() & is\_in\_category  &   Insert       \\ \hline 

    \end{tabular}
\end{center}
