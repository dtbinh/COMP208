\todo{copy/paste from github wiki and notebook, scan diagrams in notebook 2}
\section{Summary}
All client-client communication is mediated by the server. When one client
wishes to send a message to another it encrypts the message with the public key
associated with the recipient and uploads it to the server. When one client
wishes to recieve a message it downloads all new messages from the server and
parses those it can decrypt. This is performed in order to hide who recieves a
message. All messages except CLAIM messages are encrypted.

\section{Message Formatting}
\subsection{Unecrypted Messages}
Messages have a command (or type), which specifies the nature of the message;
messages have content, which specifies the details of the message;
messages have an RSA signature, which authenticates the message;
messages have a timestamp, which dates the message down to the millisecond, the
time format is unix time in milliseconds.

Messages are represented external to the system as ascii strings, and internally
via the Message class. The string representation is as follows:\\

\begin{center}
\textless \textit{command}\textgreater\textbackslash
\textless \textit{signature}\textgreater\textbackslash
\textless \textit{content}\textgreater\textbackslash
\textless \textit{timestamp}\textgreater
\end{center}

Backslashes are literal, angle brackets denote placeholder values where data
specific to a message is placed.

An example follows:
\begin{center}
POST\textbackslash\textless\textit{signature}\textgreater\textbackslash Hello, World!\textbackslash 1393407435547
\end{center}

backslashes in message content are escaped with a backslash, signatures are
base64 encoded SHA256/RSA signatures. All text is encoded in UTF-8.

\missingfigure{Diagram from wiki page}

\subsection{Encrypted Messages}
Encrypted messages contain the AES IV's; the RSA encrypted AES key; and the AES
encrypted message.

Messages are encrypted by encoding the entire message to be sent with UTF-8;
encrypting the message with a randomly generated AES key; encrypting the AES key
with RSA; encoding the RSA encrypted AES key in base64; encoding the (random)
AES initialization vectors in base64 and concatenating these three parts with a
backslash between each. The format follows:

\begin{center}
\textless \textit{AES IV}\textgreater\textbackslash
\textless \textit{RSA encrypted random AES key}\textgreater\textbackslash
\textless \textit{AES encrypted message}\textgreater
\end{center}

Backslashes are literal, angle brackets denote placeholder values where data
specific to a message is placed.

\missingfigure{Diagram from wiki page}

\section{Claiming a Username}
Each user (keypair) should claim one username. Uniqueness is enforced by the
server, and so not relied upon at all. Usernames are useful because public keys
are not human readable. In order to claim a username, one must sent an
unencrypted CLAIM message to the server. The format follows:

\begin{center}
CLAIM\textbackslash
\textless \textit{signature}\textgreater\textbackslash
\textless \textit{username}\textgreater\textbackslash
\textless \textit{timestamp}\textgreater
\end{center}

\section{Revoking a Key}
If a users private key should be leaked, then they must be able to revoke that
key. This is done by sending a REVOKE message to the server. All content signed
by the private key after the stated time will be flagged as untrusted. The
format follows:

\begin{center}
REVOKE\textbackslash
\textless \textit{signature}\textgreater\textbackslash
\textless \textit{time}\textgreater\textbackslash
\textless \textit{timestamp}\textgreater
\end{center}

\section{Profile Data}
Users may wish to share personal details with certain people, they may share
this information via profile data. Profile data is shared using PDATA messages.
A PDATA message contains a list of fields, followed by a colon, followed by the
value, followed by a semicolon. The format follows:

\begin{center}
PDATA\textbackslash
\textless \textit{signature}\textgreater\textbackslash
\textless \textit{values}\textgreater\textbackslash
\textless \textit{timestamp}\textgreater
\end{center}

The format for values follows:

\begin{center}
\textless \textit{field}\textgreater:
\textless \textit{value}\textgreater;
\ldots
\end{center}

An example follows:

\begin{center}
PDATA\textbackslash
\textless \textit{signature}\textgreater\textbackslash
name:Luke Thomas;dob:1994;\textbackslash
\textless \textit{timestamp}\textgreater
\end{center}

\section{Inter-User Realtime Chat}
Users can chat in real time

\section{Posting to own wall}

\section{Posting on anothers wall}

\section{Commenting}

\section{Liking}
