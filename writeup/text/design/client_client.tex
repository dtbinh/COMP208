\section{Summary}
\todo{consider computational cost of separating RSA header and AES message}
All client-client communication is mediated by the server. When one client
wishes to send a message to another it encrypts the message with the public key
associated with the recipient and uploads it to the server. When one client
wishes to receive a message it downloads all new messages from the server and
parses those it can decrypt. This is performed in order to hide who receives a
message. All messages except CLAIM messages are encrypted. Multiple recipients
imply multiple messages being uploaded, this is taken for granted in the text
which follows.

\section{Message Formatting}
\subsection{Unecrypted Messages}
Messages have a command (or type), which specifies the nature of the message;
messages have content, which specifies the details of the message;
messages have an RSA signature, which authenticates the message;
messages have a timestamp, which dates the message down to the millisecond, the
time format is unix time in milliseconds.

Messages are represented external to the system as utf-8 strings, and internally
via the Message class. The string representation is as follows:\\

\begin{center}
\textless \textit{command}\textgreater\textbackslash
\textless \textit{signature}\textgreater\textbackslash
\textless \textit{content}\textgreater\textbackslash
\textless \textit{timestamp}\textgreater
\end{center}

Backslashes are literal, angle brackets denote placeholder values where data
specific to a message is placed.

An example follows:
\begin{center}
POST\textbackslash\textless\textit{signature}\textgreater\textbackslash Hello, World!\textbackslash 1393407435547
\end{center}

backslashes in message content are escaped with another backslash, signatures
are base64 encoded SHA256/RSA signatures of the content of the message
concatenated with a decimal string representation of the timestamp. All text is
encoded in UTF-8.

\missingfigure{Diagram from wiki page}

\subsection{Encrypted Messages}
Encrypted messages contain the AES IV's; the RSA encrypted AES key; and the AES
encrypted message.

Messages are encrypted by encoding the entire message to be sent with UTF-8;
encrypting the message with a randomly generated AES key; encrypting the AES key
with RSA; encoding the RSA encrypted AES key in base64; encoding the (random)
AES initialization vectors in base64 and concatenating these three parts with a
backslash between each. The format follows:

\begin{center}
\textless \textit{AES IV}\textgreater\textbackslash
\textless \textit{RSA encrypted random AES key}\textgreater\textbackslash
\textless \textit{AES encrypted message}\textgreater
\end{center}

Backslashes are literal, angle brackets denote placeholder values where data
specific to a message is placed.

\missingfigure{Diagram from wiki page}

\section{Claiming a Username}
Each user (keypair) should claim one username. Uniqueness is enforced by the
server, and so not relied upon at all. Usernames are useful because public keys
are not human readable. In order to claim a username, one must sent an
unencrypted CLAIM message to the server. The format follows:

\begin{center}
CLAIM\textbackslash
\textless \textit{signature}\textgreater\textbackslash
\textless \textit{username}\textgreater\textbackslash
\textless \textit{timestamp}\textgreater
\end{center}

\section{Revoking a Key}
If a users private key should be leaked, then they must be able to revoke that
key. This is done by sending a REVOKE message to the server. All content signed
by the private key after the stated time will be flagged as untrusted. The
format follows:

\begin{center}
REVOKE\textbackslash
\textless \textit{signature}\textgreater\textbackslash
\textless \textit{time}\textgreater\textbackslash
\textless \textit{timestamp}\textgreater
\end{center}

\section{Profile Data}
Users may wish to share personal details with certain people, they may share
this information via profile data. Profile data is shared using PDATA messages.
A PDATA message contains a list of fields, followed by a colon, followed by the
value, followed by a semicolon. The format follows:

\begin{center}
PDATA\textbackslash
\textless \textit{signature}\textgreater\textbackslash
\textless \textit{values}\textgreater\textbackslash
\textless \textit{timestamp}\textgreater
\end{center}

The format for values follows:

\begin{center}
\textless \textit{field}\textgreater:
\textless \textit{value}\textgreater;
\ldots
\end{center}

An example follows:

\begin{center}
PDATA\textbackslash
\textless \textit{signature}\textgreater\textbackslash
name:Luke Thomas;dob:1994;\textbackslash
\textless \textit{timestamp}\textgreater
\end{center}

\section{Inter-User Realtime Chat}
Users can chat in in real time, this by achieved by sending a CHAT message to
all people you wish the include in the conversation. This message includes a
full list of colon delimited public keys involved in the chat. The format
follows:

\begin{center}
CHAT\textbackslash
\textless \textit{signature}\textgreater\textbackslash
\textless \textit{keys}\textgreater\textbackslash
\textless \textit{timestamp}\textgreater
\end{center}

The format for keys follows:

\begin{center}
\textless \textit{key}\textgreater:
\textless \textit{another\_key}\textgreater
\ldots
\end{center}

An example follows:

\begin{center}
CHAT\textbackslash
\textless \textit{signature}\textgreater\textbackslash
\textless \textit{key1}\textgreater:\textless \textit{key2}\textgreater\textbackslash
\textless \textit{timestamp}\textgreater
\end{center}

Following the establishment of a conversation, messages may be added to it with
PCHAT messages, the format follows:

\begin{center}
PCHAT\textbackslash
\textless \textit{signature}\textgreater\textbackslash
\textless \textit{conversation}\textgreater:\textless \textit{message}\textgreater\textbackslash
\textless \textit{timestamp}\textgreater
\end{center}

Whereby \textless \textit{conversation}\textgreater denotes the signature
present on the establishing message. An example follows:

\begin{center}
PCHAT\textbackslash
\textless \textit{signature}\textgreater\textbackslash
9f86d081884c7d659a2feaa0c55ad015a3bf4f1b2b0b822cd15d6c15b0f00a08:First!\textbackslash
\textless \textit{timestamp}\textgreater
\end{center}

\section{Posting to own wall}
When a user posts to their own wall they upload a POST message to the server of
the following format.

\begin{center}
POST\textbackslash
\textless \textit{signature}\textgreater\textbackslash
\textless \textit{message}\textgreater\textbackslash
\textless \textit{timestamp}\textgreater
\end{center}

The format of message is merely UTF-8 text, with backslashes escaped with
backslashes.

An example follows which contains the text "Hello, World!", a newline, "foo
\textbackslash bar\textbackslash baz":

\begin{center}
POST\textbackslash
\textless \textit{signature}\textgreater\textbackslash
Hello, World!\\
foo\textbackslash\textbackslash bar\textbackslash\textbackslash baz\textbackslash
\textless \textit{timestamp}\textgreater
\end{center}

\section{Posting on another users wall}
A user may request to post on a friends wall by sending them an FPOST message,
the poster may not decide who is able to view the message. The format is
identical to that of a POST message, except for the command and singular
recipient. An example follows:

\begin{center}
FPOST\textbackslash
\textless \textit{signature}\textgreater\textbackslash
Hello, World!\textbackslash
\textless \textit{timestamp}\textgreater
\end{center}

Upon receipt of an FPOST message the friend is prompted by the client to choose
whether or not to display it, and if so who may view it. Once this is done the
friend reposts the message with the command changed to POST instead of FPOST as
they would post anything to their own wall. This works because authentication is
entirely based on RSA signatures so in copying the original signature the friend
may post as the original author provided they don't alter the message (and thus
its hash and required signature).

\section{Commenting}
Commenting works similarly to posting on another's wall, so an explanation of
details of how it occurs is not provided (see prior section). The only
difference is the format of a CMNT message from an FPOST message. The format of
a CMNT message is as follows:

\begin{center}
CMNT\textbackslash
\textless \textit{signature}\textgreater\textbackslash
\textless \textit{hash}\textgreater:
\textless \textit{comment}\textgreater\textbackslash
\textless \textit{timestamp}\textgreater
\end{center}

Where \textless \textit{hash}\textgreater denotes the hash of the post or
comment being commented upon. An example comment follows:

\begin{center}
CMNT\textbackslash
\textless \textit{signature}\textgreater\textbackslash
v/sXfb3DG2qT2k2hXIH4csJy1yEG+TANRbbxQw1VkSE=:
Yeah, well, that's just like, your opinion, man.\textbackslash
\textless \textit{timestamp}\textgreater
\end{center}

\section{Liking}
Like messages are identical to comments except for the command and the the fact
that no ":\textless \textit{comment}\textgreater" follows the hash. An example
like follows:

\begin{center}
LIKE\textbackslash
\textless \textit{signature}\textgreater\textbackslash
v/sXfb3DG2qT2k2hXIH4csJy1yEG+TANRbbxQw1VkSE=\textbackslash
\textless \textit{timestamp}\textgreater
\end{center}

\section{Events}
A user may have the client remind him of an event by alerting him when it
occurs. A user may inform others of events, and they may choose to be reminded
about them. When a user creates an event just for themselves they just create a
normal event and only inform themselves of it. An event is created by posting an
EVNT message to the server. The format follows:

\begin{center}
EVNT\textbackslash
\textless \textit{signature}\textgreater\textbackslash
\textless \textit{event\_start\_time}\textgreater:
\textless \textit{event\_end\_time}\textgreater:
\textless \textit{event\_name}\textgreater\textbackslash
\textless \textit{timestamp}\textgreater
\end{center}

An example follows of a reminder for bobs birthday which occurs on the 14th of
January, the event was created on the second of January:

\begin{center}
EVNT\textbackslash
\textless \textit{signature}\textgreater\textbackslash
1389657600000:
1389744000000:
bobs birthday\textbackslash
1388676821000
\end{center}
