\todo[noline]{pretty dataflow diagrams}
\textbf{Creating an account} is done by generating an RSA keypair, and choosing
a name. An unencrypted (but signed) message is then posted to the server
associating that keypair with that name. In this way, by knowing the public key
of someone, you may discover their name in the service, but not vice versa.\\

\textbf{Connecting for the first time} Every unencrypted message stored on the
server is downloaded(signed nicknames and nothing more). At this time the local
database contains only signed messages claiming usernames. The public keys are 
not provided, these are of use only when you learn the public key behind a name. 
The rationale for not providing public keys is provided in the section regarding 
adding a friend. Messages posted after your name was claimed will require 
downloading too, as once you claim a name people may send you messages. It's 
worth noting that messages from before you connected for the first time are now 
downloaded because they can not have been sent to you (with a compliment client) 
if someone retroactively grants you permission to view something they publish it 
as a new message with an old timestamp; the sole exception to this is when you 
connect using a new device, in which case all messages since you first claimed a 
name will be downloaded.\\

\textbf{Connecting subsequently} The client requests every message stored on the
server since the last time they connected up to the present. Decryptable
messages are used to update the local DB, others are discarded.\\

\textbf{Continued connection} During a session the client requests updates from
the server every 0.5-5 seconds (configurable by the user).\\

\textbf{Adding a friend} is performed by having a friend email (or otherwise
transfer) you their public key. This is input to the client, and it finds their
username (via public posting that occurred when registering). You may now
interact with that person. They may not interact with you until they receive
your public key. Public key transferral will be performed via exchanging plaintext
base64 encoded strings, or QR codes. The user will be prompted, after retrieving
the username of the user, to categorise them.\\

\textbf{Talking with a friend or posting on your wall} is achieved by writing
a message, signing it with your private key, and encrypting one copy of it with
each of the recipients public keys before posting it to the server. The client
prevents one from posting a message to someone's public key if they have not
claimed a nickname.\\

\textbf{Posting to a friends wall, commenting and liking} may be requested by
sending a EPOST/CMNT/LIKE message to the friend (upon whose wall/post you are
posting, commenting or liking), when that friend logs in they will receive your
request, and may confirm or deny it. If they confirm then they take your (signed)
message and transmit it to each of their friends as previously described. Given
that authentication is entirely based on crypto signatures it doesn't matter
that your friend relays the message.This is required because it is impossible
for one to know who is able to see the persons wall, post, or comment upon which
you seek to post, like, or comment.
