\todo[noline]{pretty dataflow diagrams}
\textbf{Creating an account} is done by generating an RSA keypair, and choosing
a name. An unencrypted (but signed) message is then posted to the server
associating that keypair with that name. In this way, by knowing the public key
of someone, you may discover their name in the service, but not vice versa.\\

\textbf{Connecting for the first time} Every unencrypted message stored on the
server is downloaded(signed nicknames and nothing more) \footnote{clients use
bittorrent to lighten server load?} (if someone retroactivly grants you
permission to view something they publish it as a new message with an old
timestamp). At this time the local database contains only signed messages
claiming usernames. The public keys are not provided, these are of use only when
you learn the public key behind a name. The rationale for not providing public
keys is provided in the section regarding adding a friend. Messages posted
after your name was claimed will require downloading too, as once you claim a
name people may send you messages.\\

\textbf{Connecting subsequently} The client requests every message from the last
time they connected (sent by the client, not stored by the server) up to the
present. Decryptable messages are used to update the local DB, others are
discarded.\\

\textbf{Continued connection} During a session the client requests updates from
the server every 1-5 seconds (configurable by the user).\\

\textbf{Adding a friend} is performed by having a friend email (or otherwise
transfere) you their public key. This is input to the client, and it finds their
name (via public posting that occured when registering). You may now interact
with that person. They may not interact with you until they recieve your public
key.\footnote{This is the one part that will be difficult for normal users,
however any protocol by which the server stores and serves public keys is
entirely unsuitable as a MitM would be trivial on behalf of the server
operators}\\

\textbf{Talking with a friend or posting on your wall} is achieved by writing
a message, signing it with your private key, and encrypting one copy of it with
each of the recipiants public keys before posting it to the server. The client
prevents one from posting a message to someones public key, if they have not
claimed a nickname.\\

\textbf{Posting to a friends wall} may be requested by sending a specially
formated message to that friend (all handled by the GUI, like much else here),
when that friend logs in they will recieve your request to post on their wall
and may confirm or deny it. If they confirm then they take your (signed) message
and transmit it to each of their friends as previously described (authentication
is entirely based on crypto signatures, so it doesn't matter who posts the
message).\footnote{This is required because it is impossible for one to know who
their friends friends are.}
