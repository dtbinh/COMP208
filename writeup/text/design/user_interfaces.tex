As a social network, the user interface design is of high importance, as a lot 
of users of the program will have little core system knowledge, and rely 
entirely on the user interface. As a result we have looked at a variety of 
options into designing which will be the best for the project.

\section{Swing}
Swing is the primary Java GUI toolkit, providing a basic standpoint for entry 
level interface designing. Introduced back in 1996, Swing was designed to be 
an interface style that required minimal changes to the applications code, 
providing the user with a pluggable look and feel mechanism. It has been apart 
of the standard java library for over a decade, which, as I will now explain, 
may not be to our benefit.

Swing, whilst an excellent language to begin with, and write simple applications
in, is quite dated. As our group advisor put it when inquiring about what we 
would be coding the user interface in:

\begin{quote}
"You should avoid Swing to prevent it looking like it was done in the 
seventies." - Sebastian Coope
\end{quote}

Sebastian is not wrong either, as Swing does a very plain feel to it. "Fig ???" 
shows an old instant messaging system written with Swing by one of our team 
members. As you can see it is unlikely to appeal to the mass market with such 
visually plain appearance. This makes Swing, unlikely to be our GUI toolkit of 
choice, despite some of our members experience with it.

\section{Abstract Window Toolkit}
Abstract Window Toolkit (otherwise known as AWT), was another choice given 
that we are programming in Java, and synchronicity between the two would be an 
advantage. Whilst AWT retained some advantages such as its style blending in 
with each operating system it runs on, it is even older than Swing being Java's 
original toolkit, as per such making it redundant for this project.

\section{Standard Widget Toolkit}
Standard Widget Toolkit (otherwise known as SWT), is one of the more promising 
candidates so far given its look and up-to-date support packages. The latest 
stable release of SWT was only last year, and is capable of producing programs 
with a modern and professionally built appearance, as shown in "Fig ???".

Unlike both Swing and AWT, SWT is not provided by Sun Microsystems as a part of 
the Java platform. It is now provided and maintained by the Eclipse Foundation, 
and provided as a part of their widely used Eclipse IDE, something a lot of the 
team is familiar with.

\section{GWT}
GWT allows you to create HTML/Javascript based user interfaces for Java 
applications running locally. The interface is programmed in Java and then GWT 
creates valid HTML/Javascript automatically. A web server is required in order
for Javascript events to be sent to the Java application.

The user can then interact with the system by pointing their web browser at 
localhost. This has the benefit of being familiar to novice users as most modern 
computer interaction is done within a web browser. 

Another advantage of using GWT is the ability to alter the appearance of web 
pages using CSS. This facilitates the creation of a modern, attractive user 
interface that integrates nicely with current operating systems and software.

\section{Javascript}
It is possible to create the entire client application in Javascript and use a 
HTML/Javascript GUI. This approach removes the need for a local web server 
meaning the only software the user is required to run is a modern web browser.

Another advantage would be tight integration between the logic and interface 
elements of the client application and no risk of errors caused by using 
multiple programming languages.

The main disadvantage of this approach is the difficulty in implementing the 
required security measures and encryption in Javascript. This can be remedied by 
using a Javascript library such as the Forge project which implements many 
cryptography methods.

\begin{comment}
Starts to get more into client side logic here but I think it is important to 
mention as this method would require an entire rewrite of the client. -Louis-
\end{comment}