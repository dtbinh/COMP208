\section{Overview of Protocol}
Tor is an implementation of onion routing, it routes traffic from your computer
through a number of other nodes; the final 'exit' node the routes the traffic
to the final destination\cite{torProtocol}. Node IP's are listed publicly in
directory servers. In this manner the IP of clients connecting to a server is
obscured from that server.

RSA/AES is used to ensure that only you, the exit node, and the final
destination see the plaintext traffic being routed. With the use of TLS, SSL, or
other end-to-end encryption those who see the plaintext can be reduced to you
and the final destination. However a malicious exit node can MitM SSL connections
using ssl-strip or a similar tool. There are methods of avoiding this, but it is
a serious issue because users believe that SSL is secure. This exploit is found
in the wild\cite{badRelays}, and so is most definitely a concern.

Tor also supports 'hidden services' which seek to conceal the IP of the client
from the server, and the IP of the server from the client. These are
significantly more secure as the traffic never exits the tor network, however
provide no protection from the adversary as will be described later; after all,
we're assuming the server operators are colluding, so they will provide data
required for traffic confirmation.

\section{Security}
Given that Tor is a low-latency network, traffic can easily be correlated. This
problem is ameliorated in high-latency networks such as mix nets, but not
eliminated.

Tor does not seek to protect against size correlation, or time correlation of
traffic. Rather the purview of tor is to conceal the IP address of a client from
the servers which it connects to.

Should a global passive adversary have perfect visibility of the internet, they
would be able to track tor traffic from source to host by correlating the size
and time of transmissions.

\begin{quote}
The Tor design doesn't try to protect against an attacker who can see or measure
both traffic going into the Tor network and also traffic coming out of the Tor
network\cite{torOneCell}. - Roger Dingledine
\end{quote}

We can safely assume that the adversary has access to the clients traffic, since
our threat model is that of a nation state seeking to spy on its citizens.
Furthermore we may assume that the adversary has access to the content host, as
our threat model assumes that service operators may be pressured legally or
otherwise into spying on their users. Therefore we must conclude, at least for
\textit{the} adversary, that Tor is unsuitable for concealing activity in
traditional social networks, due to traffic confirmation.

Does this then mean that Tor is insecure? No. So far as we know\cite{torStinks}
the US does not currently have the ability to reliably and consistently track
tor users; if the US is incapable of doing so, it is reasonable to assume that
no other nation state has this ability. This is however not something which
should be relied upon, as assumptions widely lead to mistakes. We shall 
therefore consider Tor as unsuitable for transmitting our data, at least if we 
were to do so as a traditional social network.

\begin{quote}
With manual analysis we can de-anonymize a \uline{very small fraction} of Tor
users, however, \uline{no} success de-anonymizing a user in response to a TOPI
request/on demand\cite{torStinks}.
\end{quote}
