When designing a system in which security is a significant aspect, it helps to
define clearly exactly what adversaries are anticipated. In this section we will
describe a hypothetical adversary (hereafter 'the adversary') against whom we
will protect our users.

The adversary will be granted all powers available to all conceivable attackers,
such that no collusion of attackers may overcome our security (should it work
for any given considered attacker).

The following individual attackers are considered, those attackers excluded are
excluded on the basis that their abilities are a subset of the union of the
abilities of the already considered attackers.

\begin{itemize}
    \item Nation state without regard for international law and convention (e.g.: USA)
        \begin{itemize}
            \item Pressure those it claims governance over into doing as it
                  demands
            \item Pressure companies operating within it into colluding in an
                  attack
            \item Identify all people connecting to the server. (Formed from the
                  union of powers of the ISP and the server owner and operators)
        \end{itemize}
    \item ISP (e.g.: BSkyB)
        \begin{itemize}
            \item View all traffic on their network, after the point at which a
                  user comes under suspicion.
            \item Manipulate all traffic on their network however they desire.
            \item Identify an IP address (during a specific time) with a person.
        \end{itemize}
    \item Server Owners and Operators (i.e.: Those who own and operate Turtlenet)
        \begin{itemize}
            \item Alter the source of the server in any way they desire.
            \item Log all traffic before and after a user comes under attack.
            \item Manipulate all traffic in any way they desire.
            \item Collect the IP of all connecting users.
        \end{itemize}
\end{itemize}

Some of these claims may seem extreme, but given that companies such as BT,
Vodafone Cable, Verizon, Level 3, and others have provided unlimited access to
their networks\cite{tempora} to governmental spy agencies, we feel it is a
reasonable threat model in light of recent revelations\cite{wikiDisclosures}.

Given that our system is intended to both protect people from the governments
which claim governance over them, and mere greedy companies looking to sell or
collect user data for profit, we will assume the worst case: i.e. that all
our users, their ISPs, and the owners and operators of the Turtlenet server they
use are able to be pressured by the adversary.

We grant the adversary all the powers listed above, and assume that
all ISPs, companies, and Turtlenet server operators are actively working against
all of our users. In summary, we consider the adversary to be:

\begin{quote}
\centering
A nation state for which money is no object, claims governance over the user,
and has the ability to pressure service providers into spying on their users.
\end{quote}

\section{Scope}
We do not attempt to protect against an adversary who has access to and the
ability to modify the users hardware, nor do we attempt to conceal that an IP
uploads data to the network.

While we recognise that the ability to post messages anonymously is important,
especially considering that countries normally considered benign are prosecuting
people over whom they claim governance for saying 'offensive' things
\cite{illegalOpinions}, it is unfortunately out of scope for this project.
