A project of this scope has a rather specific target in sight. Due to its 
encrypted nature, Turtlenet can act as a form of anonymity between users who 
would otherwise be targeted by governments and/or institutions opposed to 
them. Countries such as China\cite{chinafirewall} and a majority of the middle 
east\cite{libyaEgypt} have recently seen negative press due to their 
persecution of individuals whom disagree with the ruling regime, such software 
would allow said individuals safety from what the wider world views as 
acceptable. 

Large multinational defence corporations (e.g. IBM, Thales, BAE) might also 
find Turtlenet useful, as it would allow for a secure communication tool 
between employees in an office. It could also potentially be used outside a 
company firewall to send messages securely between offices across much larger 
distances. Corporations such as defence contractors often hold security in the 
highest regard, and such a project would match their needs well.

A more likely recipient of this system however, is society itself, as we
have decided to waive our copyright granted monopoly. Should another group
decide to embark on a similar project, they will have access to this project,
to act as a baseline for their own work. See Appendix \ref{licence}.
