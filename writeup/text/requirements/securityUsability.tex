/section{Security and Usability}
There is room for improvement when it comes to increasing the level of usability 
and user-friendliness in the Turtlenet software. 

Firstly, as developers, we need to pay attention to the target audience, especially 
when it alludes to the general public, which have different ranges of user experience 
and knowledge when using a social network program. Paying attention and understand 
to how the general public thinks and the level of their knowledge are, we as developers 
can shape the program into something that is user-friendly and avoid mistakes which 
can be easily made by the users if the program is confusing. We have to be aware that 
since Turtlenet is a social media which is more security oriented, there are some 
functionalities that the general public might not be knowledgeable about.

In order to help the users who uses Turtlenet for the first time in our future 
development, is the use of 'pop up manual'. These instructions will pop out in a 
subtly manner in specific areas on the website when the user uses Turtlenet for 
the first time. An example is under the public key field. The general public would 
have no idea what a public key is or what is to do with it. A pop up manual is useful 
for this case, it will pop out saying in a generic, non-jargon or technical sense 
saying "Public key is the key to share with your trusted friends! Hence its name, it 
behaves like a key which can be passed along to others so they can unlock and view 
your profile."

Lastly, in order to make future improvements of the program, one important and effective 
way to do so is to listen to the customers' feedback. There are several platforms to 
gather feedback from customers, one common way to do it nowadays is using social media 
websites, and Twitter will be the best option for this case. Users can easily mention us 
in tweets, could be anything from actual feedback, rants, questions regarding the use of 
the program, what they like about it, and positive feedback. This could become useful to 
us developers to see how the customer think, feel and behave towards Turtlenet, and 
possibly the trends about their use of it. Using such information, developers can improve 
on certain areas of the program according to the users needs. Users are the one who uses 
the program regularly and know what is best when it comes to functionality. 