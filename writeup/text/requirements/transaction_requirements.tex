Due to the nature of Turtlenet there may exist no central database, rather each
client maintains their own local database of everything they know. The data
forming this is all stored centrally, however to build a complete database would
require the private key of every user of the service, which clearly we do not
have access to.

There are 3 categories of data transaction:
\begin{itemize}
\item 1. Data entry
\item 2. Data update and deletion
\item 3. Data queries
\end{itemize}

\section{Profile creation of the user}
All that is required to use a Turtlenet server is a valid RSA keypair. Users
don't have accounts per se, but rather associate profile data with a public key
if they so desire. Users have no login information, rather posts are
authenticated via RSA signatures. Usernames are the sole public information in
our system, and as such each client has a complete list of usernames.

When a client first connects it is advisable, albeit not required, to claim a
username. This is done merely by posting that username, and a signed hash of it
to the server. Therefore the DB must store all such CLAIM messages.

Optional profile data which the user may enter is stored as PDATA messages, and
the databse will be required to store these.

\section{Adding of user relations}
Communication between people on Turtlenet requires that one is in possesion of
the public key of the recipient, and should they wish to respond then they must
be in possesion of your public key. We define 'A being related to B' to mean
that A is in possesion of B's public key, and B is in possesion of A's public
key. This is given a special name as it is a very common situation.

A user may be uniquely identified by their public key, and it may be used to
derive their username, if they claimed one. Being in a relation with someone
doesn't mean that you can see any profile information of theirs, however the
GUI will ask the user whether they wish to share their own profile information
with someone when they add that persons key.

\section{Assigning relations into categories}
When a user adds a relation, he has a choice of adding him into a specific
category (or categories). A user can create any category he wants by going to
the options and click 'Add new category'. The database then records the new
category into the category table.  The user then can then assign the relation
into the existing category.

Categories are useful because they allow the user to share their posts with a
predetermined set of people automatically, withing having to list each
individual as a recipient.

\section{Adding of posts}
Users may post on their, or - with permission - others, walls. A post has a list
of people who can see it (the user may choose a previously defined category or
a specific list of people) however this list isn't public so only the DB of the
author of a post (and the owner of the wall) will contain information as to who
is able to see it.

The post itself has a timestamp, a signature (authenticating it), and content.
The databse will store all of these.

NB: When a user posts something, they are automatically added to the list of
recipients. A users own posts are downloaded from the server, just like everyone
elses, and are in no way special.

\section{Adding of events}
The database will store events, these may be created by the user, or recieved
from other users. At the appropriate time the GUI will notify the user of an
event occuring. Example include birthdays, deadlines, and important dates.
Events recieved from other users must be accepted by the user before the GUI
will alert the user of them, for this reason the DB must also store whether an
event is accepted or not.

\section{User creating a new message}
A user can initiate a conversation with (an)other user(s) by creating a new
message. Messages are merely a special case of wall posts, which are handled
differently by the GUI.

\section{Receiving Content}
When the client connects it will download all messages posted since it last
connected, it will then attempt to decrypt them all using the users private key.
Those messages which are successfully decrypted are authenticated by verfying
the signature and the content added to the database. It is in this manner that
all content is passed from server to client.
