An estimate is hereafter given as to the size of all stored messages, and the
amount of data which would need downloading by each client when it is started.
The following assumptions are used throughout:
\begin{itemize}
\item A users average message posted to their wall is ~200 characters
\item A users average number of messages posted to their wall per day is 10
\item A users average number of friends is 100 (each and every friend represents
one key exchange)
\item A users average private message (to single user) is 50 characters
\item A users average number of private (to single user) messages per day is 300
\end{itemize}
With these generous estimates, each user would generate (200*10*100)+(50*300*1)
bytes of raw data per day. Assuming a 10\% protocol overhead we would see
236,500 bytes of data per day per user.\\

The storage space required for a server is therefore 86MB per year per user. On a
server with 50,000 users that has been running for 3 years, there would be just
1.3TB of data.\\

Every time a client connects, it must download all messages posted since it last
connected to the server. To mitigate this we may run as a daemon on linux, or a
background process in windows, that starts when the user logs in. If we can
expect a computer to be turned on for just 4 hours a day then 20 hours of data
must be downloaded. ((236,500*no\_of\_users)/24)*hours\_off\_per\_day bytes must
be downloaded when the users computer is turned on.

The following table shows the delays between the computer turning on, and every
message having been downloaded (assuming a download speed of 500KB/second, and a
network of 1000 users).

\begin{table}[h]
    \centering
    \begin{tabular}{lc}
    Hours off per day & Minutes to sync       \\ \hline
    0                 &  0 \\
    4                 &  1.3\\
    10                &  3.2\\
    12                &  3.9\\
    16                &  5.2\\
    20                &  6.5\\
    \end{tabular}
    \caption{Hours a computer is turned off per day vs minutes to sync}
\end{table}

We feel that waiting 2-5 minutes is an acceptable delay for the degree of privacy
provided. Once the user is synced after turning their computer on, no further
delays will be incurred until the computer is shut down.

Due to the inherently limited network size (\textless1500 users of one server is
practical) we recommend a number of smaller servers, each serving either a
geographic location, or a specific interest group.

While this latency could be avoided, and huge networks (\textgreater1,000,000)
used, it would come at the cost of the server operator being able to learn that
somebody is sending or receiving messages, and also who those messages are sent
to/from (although they couldn't know what the messages said).

The server therefore merely needs a fast internet connection to upload and
download content from clients. The client is required to perform a significant
amount of encryption and decryption, however the client will almost certainly be
able to encrypt/decrypt faster than a connection to the internet so the network
speed may be considered the limiting factor for users on the internet
\cite{cryptoSpeed}. Large companies however may very well use the system over a
LAN, however these can be reasonably expected to have fairly modern computers
which can more than handle RSA decryption.
