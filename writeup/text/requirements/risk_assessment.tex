\excludecomment{comment}
% ^^^ doesn't show comment blocks on exported (PDF) files

\section{Parallel Tasks}
\paragraph{}

A big concern for any project is the amount of tasks that will be performed
simultaneously \cite{OracleWP}.
For every task that is carried out together, but potentially
separate from each other, risk is increased - with more tasks making a more
dramatic increase of potential failure for the project.  For the planning
section of this project we have performed a large amount of tasks simultaneously
which may be detrimental to our quality of work later in the project.

In order to reduce or even eliminate the risk of too many parallel activities,
the project should be planned using Gantt and PERT charts to reduce the amount
of tasks being performed simultaneously and have more milestones within the
project.  This will help effectively split up the project into more manageable
sections which will not only make the project seem simpler to complete but will
improve the monitoring capabilities of the project as well.

\section{Group Work}
\paragraph{}

Working within a group can make deliverable dates difficult to achieve.  This
can be due to a lack of communication, unavailability of party members or an
incapability to meet deadlines for some of the members.  A meeting of minds also
includes an assemblage of work ethics.  Because of this, work may grind to a
halt as members argue over personal yet trivial matters such as formatting
documents or a varying opinion on what is classed as 'enough work' for a task.

Combating the disadvantages of working as a group can be difficult.  As some
problems are part of a group unable to function either properly of efficiently
together, this can be the breaking factor of the project.  This is a risk that
cannot be eliminated but can be reduced.  A way of minimising the amount of
damage that the risk will do would be to have a centralised form of contacting
members of the group - examples being a website or using a revision control
system such as 'Apache Subversion 'or 'Git', will give a common area for the
group to look for potential absences or reasons for reduce work output from
members. 

The best way to reduce the risk of differing qualities of work between the group
would be to define a standard of work between the group - such as the layout of
source files in programming languages or a house style for formal documents as
part of the group's external identity.  Having this be available to the group in
some form, such as in a text file within a shared area will allow the group to
refresh their memories of parts of the set standard that they wouldn't follow
otherwise.

\section{Deadlines}
\paragraph{}

Deadlines are the final day or dates that an object needs to be completed by.
Sometimes within a project the deadline may be overstepped due to any of the
risks mentioned within this document, which can lead to something small such as
being berated by the project leader or something serious such as a breach in
contract with the client.  For these reasons, deadlines need to be adhered to
so that the project can continue on schedule.

Reducing the risk of deadlines are important, especially for those that are not
capable of monitoring their time effectively.  By providing deadlines as a range
of dates as opposed to a singular date, there is increased flexibility within
the project and it gives some people more time to finish their work if it is
required.  By using a range of dates the group can finish on the beginning of
the deadline range - a sort of pseudo-deadline - meet up and discuss whether
alterations need to be made on the work and then use the remainder of the time
until the end of the deadline range to perform them.

\section{Scope}
\paragraph{}

The scope is what the project will be encompassing and therefore is one of the
most important sections as it defines what you'll be doing for the entirety of
the project.  That's not the only risk associated with the scope 
\cite{RiskList}.
There is also scope creep, which is when the scope grows to cover more work than
the project originally intended, often without an increase in resources matching
the higher load on the project deliverables.  Performing estimates on the scope,
as well as anything else in the project, can be inaccurate as you are
essentially guessing the near future which is difficult at the best of times.

To minimise the risk placed upon the scope, it is best to define what exactly is
required of the project before any work takes place on the deliverables.  For
example it is best to define an encryption method for a project at the beginning
and sticking to it rather than changing the method which may require a different
implementation, creating more work.  If at a later phase ambiguities appear in
the scope, a meeting to define or even redefine these points should occur before
any more work is carried out on the offending article, reducing the amount of
change to the project that shall occur.

\section{Change Management}
\paragraph{}

Change Management is the application of a structured process and set of tools
for leading the people side of change to achieve a desired outcome
\cite{changeManDef}.  Problems that are associated with Change Management
include conflicts which occur between stakeholders, as they may be
disagreeing in how the project should move forward, an assumption that an
irreparable state has befell the project due to a drastic amount of changes that
have been placed upon the project or even ambiguous or inaccurate changes being
added onto the project \cite{RiskList}.  All of these can amount into an
increase in workload or a decrease if the targets haven't been defined properly.

To reduce the amount of risk involved with Change Management, communication and
clear definition on what the project needs to perform is required.  Stakeholders
should be as detailed as possible at every stage so that no ambiguity is caused,
or cleared up if any does occur.

\section{Stakeholders}
\paragraph{}

Stakeholders are people that have an interest in the project, whether they are
the members of the group, the group's monitor/superior or the target audience of
the project.  Some of the problems that Stakeholders cause for the project
members include losing interest - if they become uninterested with the project
then they may back out, which can be dangerous for the project if they were
providing any form of input, such as experience in the target field or economic
support.  Most Stakeholders are disillusioned when it comes to the project - 
they are unaware of what the deliverables will be or have a twisted view on what
and how the final product will perform its intended purpose.  As always there is
also a risk in terms of quality - Stakeholders may give ambiguous input both
accidentally or on purpose, depending whether the Stakeholder wants the project
to fail or not \cite{RiskList}.

The best way to reduce the amount of risk involved with Stakeholders would be to
keep them informed of the project's current status through external
communication such as e-mail and through meetings so that the team can
personally inform the Stakeholder with relevant information which should ease
their mind of any apprehensive thoughts about the project \cite{stakeInfo}.

\section{Platforms}
\paragraph{}

The main risk in Platforms would be the difference between the chosen
development platform and the target market's system.  The change between
executable files for different operating systems are usually great enough so
that a separate executable is required for each distinct operating system.
What may also cause problems, especially with low-level programming, would be
differing architectures, hardware sets and how the system reads commands
\cite{wikiArchDiff}.
Another problem with platforms would be whether the required software for the
project is installed, such as any required run-time environments or files
which are needed to use Structured Query Language databases.

This risk can be eliminated if platform-independent code is used - such as the
Java Programming Language \cite{javaFeatures}.  This would mean that no changes
in implementation would be needed and database functionality could occur within
the platform-independent environment if need be.  Otherwise to reduce the amount
of risk involved with the varying systems that the target audience may own,
compiling the source on different virtual systems to create executables for the
many various platforms available would suffice.  Of course, this can be
mitigated by choosing to not support other systems in favour of only allowing
the development platform and its Operating System to be supported.

\section{Integration}
\paragraph{}

The integration of the project can be high risk due to a couple of factors:

\begin{itemize}
    \item The intended environment is incompatible or unavailable
    \item Incomplete testing means the final product may be buggy
    \item Final product doesn't work (e.g. bad link to database)
    \item Product lowers efficiency due to learning curve \cite{RiskList}
\end{itemize}

In order to combat the risks involved in implementation, having a set testing
day in an isolated environment can allow the completed builds of the project to
be evaluated before being given to the target audience.  This will allow the
checking of compatibility with the system as well as in-house bug testing.  A
manual or help section could be implemented into the system so that the learning
curve is not as steep compared to not having such resources.

\section{Requirements}
\paragraph{}

Requirements are not just a list of functional needs and wants but also the
constraints on the project as well.  However, their are similar risks involved
in the requirements, such as generalisation, ambiguity or even being incomplete.
Another risk to do with requirements is whether they align with the design
factor or not.

An example would be having both 'fast processing' and 'system
independence' as requirements - C++ is faster but Java is independent of
platform and although speed may not be an issue with smaller data, larger chunks
of data will undoubtedly have an effect on interpreted code
\cite{javaCbenchmark}.

To minimise the risk with requirements, communication between group members and
stakeholders is needed - making sure that the requirements and the scope are in
line with each other and that any suggested changes are properly handled with
little to no ambiguity.  Choosing a design structure and sticking to it is also
beneficial to the project.  Reducing the workload of the implementation can
help towards minimising the risks of requirements and the program, such as
removing old data that is no longer needed upon the program's start-up.

\section{Authority}
\paragraph{}

Without distinct authority within the project, risks can become apparent.
If the members of the project do not have the correct privileges on the target
system to perform what is required, work output slows or even stops until the
matter is resolved.  Another risk would be misguided authority - where the team
is unclear who has been given the authority to perform a task and therefore 
there are multiple members allocating the same task to themselves, which will
slow down the efficiency of the team due to duplicated work.

Lowering the negative impact of Authority is done through the use of clear
definitions.  Allocating work to project members and centralising a form of
'to-do' list so that project members can look up what has been assigned to them.
Another way of reducing the amount of inefficiency caused by problems with
authority would be to make sure the permissions are correctly set up on both the
testing and target systems.

\section{External}
\paragraph{}

There are a couple of external factors which may impact the project in a
negative manner.  The first being any legal restrictions.  This is important as
there is a chance that the final product may be used in a location that differs
to the geographical area that it was developed in.  For example there is a law
within the UK called the 'Key Disclosure Law' which means that you must give
decryption keys to UK authorities \cite{ukCryptLaw}.  In the United States
however, it is something of a grey-area, as giving up encryption keys could
violate the fifth amendment, as doing so could give incriminating evidence
against yourself:

\begin{quote}
'unlike surrendering a key, disclosing a password reveals the contents of one’s mind and
is therefore testimonial.' \cite{usCryptLaw}
\end{quote}

\todo{cite RIPA part 3 from legislation.gov}

Not only is the law a big risk in projects, but also nature.  If you are
situated where natural disasters can happen or otherwise things such as heavy
weather occur, this can reduce the work flow by denying the team members access
to their workspace.  Another factor that is external is the changing of
technology.  Updates to programming languages can lead to deprecated functions
or newer operating systems may not be capable of running the same software as
their previous iterations, meaning an increased amount of work to keep the
software compatible with the target system.

Reducing the amount of risk caused by external factors is difficult as the
project team have little to no influence upon them.  For example the team cannot
bypass any laws that govern the area that the program will be used in so they
must be adhered to as part of the constraints of the project.  Natural disaster
cannot be stopped but if you are able to, bringing some of the work back so you
could work on it during bad weather may reduce the impact that said weather will
have on the project.  To reduce the damage caused by software deprecation it is
ideal if the functionality coded in the project is not old, or otherwise buggy,
so that maintaining or updating the software will require less work.

\section{Project Management}
\paragraph{}

Project Management, or rather a lack of, can also be a risk to the endeavours of
the team.  If the group has been asked to reduce or combine the amount of stages
in the System Development Life Cycle (SDLC), this can increase the risk of the
project failing because it leaves more room for error - combining the stages
will often cause a decrease in quality as less resources are being dedicated to
a particular section of the project.  A lack of Project Management will also be
seen as a high risk because of how difficult it is to monitor a project and its
success without these tools.

To reduce the risk that Project Management will apply upon the project, a formal
methodology, such as the 'waterfall' method could be implemented.  This would
however reduce inefficiency as the output needs to be moderated and cleared
before the start of the next stage in the SDLC can occur.  On the other hand an
informal methodology would increase the risks but may potentially allow the
project to be completed within a smaller time frame and to the same standard.

\section{User Acceptance}
\paragraph{}

Just because a project has been made for a target audience doesn't mean that
\textit{that} audience will like it.  During testing the target market may
reject the initial builds of the project due to the way it does or does not work
or the look of the project could mean that it is unwieldy to use, whether it is
due to low quality or the interface being anti-intuitive.

The main method of reducing the risk pre-emptively is to perform research on any
currently available software that achieve similar goals to the project's.  By
doing this you can find out what users are acquainted with and create a similar
yet unique design or use the competitors as a way of highlighting what is wrong
with the current market and create something entirely different.  Another method
which does require more work is to take in user feedback during testing and
implement their suggestions for the look of the project, or the inner mechanics
if they have the knowledge to suggest improvements.

\section{Conclusion}
\paragraph{}

In order to reduce the risk of the project as a generalisation, it is suggested
that you:

\begin{itemize}
    \item Have a centralised communication system used by all members - this
          reduces all communicative related risks.
    \item Define team objectives and allocation clearly - this reduces the
          authority-based risks as well as any that are communicative.
    \item Define a target system for development - other types of platform can
          be supported at a later date should the need arise.
    \item Create and uphold a work ethic to be followed by everyone - this helps
          to maintain a standard of quality throughout the project.
    \item Testing should be first on each individual module/deliverable, then as
          a whole.  This improves bug catching and helps monitor the quality of
          the project.
    \item Choose a methodology and follow it - this creates a standard of work
          ethics which will give a layout as well as structure to the project.
\end{itemize}

By following these pointers a moderate amount of risk can be mitigated with
little need for concern.  Do note that the legality of the project in differing
countries should be researched and followed, should the project be in use within
that country.
