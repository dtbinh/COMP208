\begin{comment}

Resources used in the production of this file:
1 - The Benefits of Risk Assessment for Projects, Portfolios, and Businesses
    An Oracle White Paper June 2009    http://www.oracle.com/us/042743.pdf
2 - 130 Project Risks (List)
    posted by Anna Mar, Simplicable, March 11, 2013 
    http://management.simplicable.com/management/new/130-project-risks

\end{comment}

\excludecomment{comment}

\section{Parallel Tasks}
\paragraph{}
\cite{OracleWP}
A big concern for any project is the amount of tasks that will be performed
simultaneously.  For every task that is carried out together, but potentially
separate from each other, risk is increased - with more tasks making a more
dramatic increase of potential failure for the project.  For the planning
section of this project we have performed a large amount of tasks simultaneously
which may be detrimental to our quality of work later in the project.

\paragraph{}

In order to reduce or even eliminate the risk of too many parallel activities,
the project should be planned using Gantt and PERT charts to reduce the amount
of tasks being performed simultaneously and have more milestones within the
project.  This will help effectively split up the project into more manageable
sections which will not only make the project seem simpler to complete but will
improve the monitoring capabilities of the project as well.

\section{Group Work}
\paragraph{}

Working within a group can make deliverable dates difficult to achieve.  This
can be due to a lack of communication, unavailability of party members or an
incapability to meet deadlines for some of the members.  A meeting of minds also
includes an assemblage of work ethics.  Because of this, work may grind to a
halt as members argue over personal yet trivial matters such as formatting
documents or a varying opinion on what is classed as 'enough work' for a task.

\pagebreak

\begin{comment}
Page formatting makes the file look more professional with a page break above.
May be subject to change when this file is absorbed into the amalgamated file.
\end{comment}

\paragraph{}
Combating the disadvantages of working as a group can be difficult.  As some
problems are part of a group unable to function either properly of efficiently
together, this can be the breaking factor of the project.  This is a risk that
cannot be eliminated but can be reduced.  A way of minimising the amount of
damage that the risk will do would be to have a centralised form of contacting
members of the group - examples being a website or using a revision control
system such as 'Apache Subversion 'or 'Git', will give a common area for the
group to look for potential absences or reasons for reduce work output from
members. 

\subparagraph{}

The best way to reduce the risk of differing qualities of work between the group
would be to define a standard of work between the group - such as the layout of
source files in programming languages or a house style for formal documents as
part of the group's external identity.  Having this be available to the group in
some form, such as in a text file within a shared area will allow the group to
refresh their memories of parts of the set standard that they wouldn't follow
otherwise.

\section{Deadlines}
\paragraph{}

Deadlines are the final day or dates that an object needs to be completed by.
Sometimes within a project the deadline may be overstepped due to any of the
risks mentioned within this document, which can lead to something small such as
being berated by the project leader or something serious such as a breach in
contract with the client.  For these reasons, deadlines need to be adhered to
so that the project can continue on schedule.

\paragraph{}

Reducing the risk of deadlines are important, especially for those that are not
capable of monitoring their time effectively.  By providing deadlines as a range
of dates as opposed to a singular date, there is increased flexibility within
the project and it gives some people more time to finish their work if it is
required.  By using a range of dates the group can finish on the beginning of
the deadline range - a sort of pseudo-deadline - meet up and discuss whether
alterations need to be made on the work and then use the remainder of the time
until the end of the deadline range to perform them.

\section{Scope}
\paragraph{}
