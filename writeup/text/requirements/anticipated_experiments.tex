\section{Performance Testing}
Evaluating how well the system performs under a high work load.
\begin{itemize}
\item Test to see how many simultaneous clients the server can handle.
\item Test to see if the data received from the server under a high work load is 
accurate.
\item Test the impact of a large number of clients on the servers response time.
\end{itemize}
A high work load will be simulated by automated clients performing user actions
at random. The server should be capable of allowing these clients to
communicate with one another quickly. The maximum number of concurrent clients 
possible without noticeable lag (1 second) should be recorded.

\section{Robustness Testing}
System level black box testing.  
\begin{itemize}
\item Devise a series of inputs and expected outputs.
\item Run these inputs through the system and record the actual outputs.
\item Compare the actual outputs with the expected outputs.
\end{itemize}
Inputs used should range from expected use patterns to silly as users tend to do 
things totally unexpected. Expected and actual outputs should be recorded in a 
spreadsheet for easy comparison. Any differences will indicate problems with the
system which need to be fixed.

\section{Recoverability Testing}
Evaluating how well the application recovers from crashes and errors.
\begin{itemize}
\item Restart the computer while the application is running.
Ensure the local database is not corrupted.
\item While the application is running terminate the computers network 
connection. Ensure the application continues working after the connection is
re-established.
\item Send a badly formatted message to another client. Ensure the application 
is able to keep running after receiving unexpected data from another client.
\end{itemize}
Each test should be run several times. If any test fails once or more this 
indicates that the system is bad at recovering from crashes and/or failures.
In the case of a failure changes to the system should be implemented to improve
recoverability.

\section{Learnability Testing}
Trialling the user interface with non expert users. Users should be able to use
the system with minimal frustration and, ideally, without consulting the manual.
\begin{itemize}
\item Ensure users understand how to add friends, send messages, create posts, 
comment on posts and like posts.
\item Ensure users don't spend excessive time searching for functions within the
interface.
\item Ensure error messages can be understood by the user and offer 
understandable advice on how to proceed.
\end{itemize}
Each test should be run several times with different users. If more than one 
user fails a test then changes need to be made to the interface. A single user 
experiencing problems is not an indication of a problem with the interface but 
instead suggests user incompetence.

\section{Security Testing}
The main goal of the system is to be secure. To ensure this goal is met the 
security of the system should be tested.
\begin{itemize}
\item Send non standard messages to clients. These should be rejected. If there 
is a flaw in the system the client may reveal information unintended for the 
recipient, in this case the program sending non standard messages.
\item Recruit experienced programmers from outside of the group to attempt to 
penetrate or otherwise break the system. All attempts should be unsuccessful.
\item Simulate a denial of service attack. The server should be able to recover
from the attack quickly and with minimal impact on the clients. Blocking such an
attack is beyond the scope of this project.
\end{itemize}
If any test fails this indicates a vulnerability in the system which 
should to be corrected immediately. Security tests should be rerun after any 
changes during the testing phase to ensure new vulnerabilities are not 
introduced.
