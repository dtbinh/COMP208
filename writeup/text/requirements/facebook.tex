\section{Overview}
A user has a profile with information about them, they may add other users as
'friends', friends may view each others 'posts' and talk to each other. Posts
are multimedia messages typically visible to all the friends of the person who
made the post. Most posts can be commented upon, and both posts and comments may
be 'liked'. Liking merely publically marks the fact that you approve of
something.

\section{Registration}
In order to be a user of facebook, one must register. In doing so you provide
facebook with the following information, this may also be used to later reset
the password of your account, should you forget it.

\begin{itemize}
\item First Name
\item Last Name
\item E-Mail
\item Password
\item Birthday\todo{friends of a user are automatically alerted of friends BD's}
\item Sex
\end{itemize}

In order to register one must read and agree to their terms \cite{fbterms}, read
their data use policy \cite{fbdatause}, and read their cookie policy
\cite{fbcookies}.
Given profile information can be changed at a later date, within certain bounds.
Facebook requires the use of your real name, and in fact forbids all false
personal information, under their terms.\cite[4.1]{fbterms}

\section{Account Managment}
The user is given the ability to set the security defaults for their posts and
information. Tese options include who is able to see wall posts, whether
comments are enabled by default, and who may see which aspects of your profile
information. You can also manage the permissions granted to facebook apps\todo{
more information on FB apps}.

A users profile may contain the following information:
\begin{itemize}
\item Work and education
\item PLece Lived
\item Relationship
\item Basic Information
    \begin{itemize}
        \item Birthday
        \item Relaionship
        \item Status
        \item Anniversary
        \item Languages
        \item Religious
        \item Political
        \item Family
        \item Contact Information
    \end{itemize}

\item \begin{table}[h]
    \centering
    \begin{tabular}{ll}
    field         & description\\ \hline
    photo         & all the photo have user's taged\\
    friend        & what friend the user had\\
    note          & what notes the user droped and uploaded to facebook .\\
    groups        & what group have user join.\\
    events        & what events user have\\
    likes         & what page (unknow type) user liked.\\
    apps          & what apps user have.\\
    books         & what books page user liked/follow.\\
    TV programmes & what TV page user liked/follow.\\
    films         & what films page user liked/follow\\
    music         & what music(or stars) user liked/ follow.\\
    sports        & what sport page user liked\\
    place         & where's the place that user had been .\\
    \end{tabular}
    \caption{user adds a new post}
    \end{table}
\end{itemize}

\section{Friend}
In facebook, 'friending' someone is symmetric; that is, if you are friends with
them, they are friends with you. The facebook severs store which user is friends
with which other users. Adding another user as a friend is simply a matter of
sending that user a friend request, and having it approved by the second user. A
user may see a list of all who are their 'friend' on FB, in the friend list.
After friending somebody that persons wall posts will appear on your news feed,
and you will be able to chat with that user.

In order to add friends, facebook allows you to see your friends friend lists,
and search by name, email, and location for other users. Facebook also suggests
other users whom you may already know IRL, based on your friends friends.
Non-users are also able to search facebook for people that they may know.

\section{Post}
\subsection{Posts, and functions thereof}
Facebook allows a user to post on their wall or friend's wall (if they are
friens with the facebook user). Posts may contain: text, images, videos, or any
combination thereof.

A user posting a post may do the following:
\begin{itemize}
\item delete their own post
\item rewrite their own post
\item decide who may view a post, the options are as follows:
    \begin{itemize}
    \item public
    \item private
    \item only-me
    \item friends only
    \item friends of friends
    \item \ldots \todo{does FB allow sharing to one or two specific people?}
    \end{itemize}
\end{itemize}

\subsection{Interaction with anothers posts}
A post will typically be displayed on the newsfeeds of the people who are able
to see it, due to this the name of the person who made a post is always
displayed next to it. Posts themselves may be commented upon, liked, and
reposted to the viewers wall ('shared') with an additional message; the number
and names of people who have liked a post is displayed underneath it; likes may
be cancelled at a later date. The comment function may however be disabled by
the user who makes a post.

A user may hide specific posts, or hide all posts by a specific user. They may
also, instead of hiding anothers posts alltogether, merely prevent them from
being automatically displayed on their newsfeed. \todo{See 72d5e2dc, what
is 'set a notification'?} A user may report an image, video or comment to
facebook team (for example:the post is offensive). Comments may also be liked,
hidden, and reported; following such a report FB is able to remove offensive or
illegal posts. \todo{allow user to share the post on third-party web(e.g.
YouTube, Steam information): really? I don't remember seeing this option in
steam}

Images which are posted may be tagged, this allows other users to mouse-over
parts of the image and be informed who is pictured. This functionality is also
used to add all posted images of someone to their profile.

\section{wall}
A users wall stores all the posts of the user posted since the account was
created and the informaation about the user, this information is presented in
reverse choronlogical order, so that recent events are at the top of the page
and easily visible. Other users may view the users wall by clicking the name of
the user from anywhere in facebook. Other users may post on a friends wall as
well as the owner, see section on posts for more information; In this case, both
the poster and the ownder of the wall can delete the post. Facebook also retains
the power to erase any content on its service.

Posts mentioning a user are automatically reposted to that users wall, this can
occur manually or when that person is tagged in an image.

\section{Chat System}
Facebook allows a user to chat with their friends, and will inform a user of
whether their friends are online or not (though this can be faked), and whether
the user you are chatting with has read the last message that you sent them.
Facebook determines that you have read a message when... \todo{how does FB do
this?}. You are also informed whether your friend is logged in on a mobile
device or not.

Whole groups of users may chat together, in multi-user conversations. Facebook
also supports video calling and file transfer during chats. If a user does not
wish to be bothered by another using chatting with them, then they may 'mute'
that users conversion. Users spamming via chat may be reported to facebook.
Because multi-user conversations (and indeed long running one-to-one
conversations) can get rather large, facebook allows you to hide the history of
a conversation.

Facebook chat alerts the user to new messages in a conversation by playing a
sound.

\section{Architechture}
\ldots

\section{Security}
In order to use facebook post registration a user must 'log in'. This places an
authentication cookie on the users computer which gives anyone in possesion of
it the ability to act as that user.\todo{verify they didn't change this since I
left FB} Users login using their email and password.

there might be some checking to the user. For example ,Once the user login in,
the facebook AI will check on the userIP , if the IP shows the location is to
far from last login IP , the facebook might need to confirm the user by asking
whats the friend's name on the photo that user had tap .Nevertheless, If the
user forgotten the password,ask for Email, Phone, Username or Full Name to help
the user to get the password
