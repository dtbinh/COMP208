\section{Overview}
A user has a profile with information about them, they may add other users as
'friends', friends may view each others 'posts' and talk to each other. Posts
are multimedia messages typically visible to all the friends of the person who
made the post. Most posts can be commented upon, and both posts and comments may
be 'liked'. Liking merely publicly marks the fact that you approve of
something.

\section{Registration}
In order to be a user of facebook, one must register. In doing so you provide
facebook with the following information, this may also be used to later reset
the password of your account, should you forget it.

\begin{itemize}
\item First Name
\item Last Name
\item E-Mail
\item Password
\item Birthday
\item Sex
\end{itemize}

In order to register one must read and agree to their terms \cite{fbterms}, read
their data use policy \cite{fbdatause}, and read their cookie policy
\cite{fbcookies}.
Given profile information can be changed at a later date, within certain bounds.
Facebook requires the use of your real name, and in fact forbids all false
personal information, under their terms.\cite[4.1]{fbterms}

\section{Account Management}
The user is given the ability to set the security defaults for their posts and
information. These options include who is able to see wall posts, whether
comments are enabled by default, and who may see which aspects of your profile
information. You can also manage the permissions granted to facebook apps.

Access may be gained to an account by knowing certain information, the intent is
to allow people to recover their account if they forget their password.

A users profile may contain the following information:
\begin{itemize}
\item Work and education
\item Place of Birth
\item Relationship
\item Basic Information
    \begin{itemize}
        \item Birthday
        \item Relationship
        \item Status
        \item Anniversary
        \item Languages
        \item Religious
        \item Political
        \item Family
        \item Contact Information
    \end{itemize}
\end{itemize}

\begin{table}[h]
    \centering
    \begin{tabular}{ll}
    Field         & Description\\ \hline
    Photo         & All the photos the user's has tagged\\
    Friend        & What friends the user has\\
    Note          & What notes the users up/downloaded to facebook\\
    Groups        & What groups the user has join\\
    Events        & What events user may be attending\\
    Likes         & What page(s) (unknown type) the user liked\\
    Apps          & What apps the user has\\
    Books         & What book pages the user liked/followed\\
    TV programmes & What TV pages the user liked/followed\\
    Films         & What films pages the user liked/followed\\
    Music         & What music(or stars) the user liked/followed\\
    Sports        & What sport pages the user liked\\
    Place         & Where's the user has been\\
    \end{tabular}
    \caption{The user adds a new post}
\end{table}

\section{Friend}
In facebook, 'friending' someone is symmetric; that is, if you are friends with
them, they are friends with you. The facebook severs store which user is friends
with which other users. Adding another user as a friend is simply a matter of
sending that user a friend request, and having it approved by the second user. A
user may see a list of all who are their 'friend' on FB, in the friend list.
After friending somebody that persons wall posts will appear on your news feed,
and you will be able to chat with that user.

In order to add friends, facebook allows you to see your friends friend lists,
and search by name, email, and location for other users. Facebook also suggests
other users whom you may already know IRL, based on your friends friends.
Non-users are also able to search facebook for people that they may know.

\section{Post}
\subsection{Posts, and functions thereof}
Facebook allows a user to post on their wall or friend's wall (if they are
friens with the facebook user). Posts may contain: text, images, videos, or any
combination thereof.

A user posting a post may do the following:
\begin{itemize}
\item Delete their own post
\item Rewrite their own post
\item Decide who may view a post, the options are as follows:
    \begin{itemize}
    \item Public
    \item Private
    \item Only-me
    \item Friends only
    \item Friends of friends
    \end{itemize}
\end{itemize}

\subsection{Interaction with another's posts}
A post will typically be displayed on the news feeds of the people who are able
to see it, due to this the name of the person who made a post is always
displayed next to it. Posts themselves may be commented upon, liked, and
reposted to the viewers wall ('shared') with an additional message; the number
and names of people who have liked a post is displayed underneath it; likes may
be cancelled at a later date. The comment function however, may be disabled by
the user who makes the post.

A user may hide specific posts, or hide all posts by a specific user. They may
also, instead of hiding another's posts all together, merely prevent them from
being automatically displayed on their news feed. A user may report an image,
video or comment to facebook team (e.g. the post is offensive). Comments
may also be liked, hidden, and reported; following such a report FB is able to
remove offensive or illegal posts.

Images which are posted may be tagged, this allows other users to mouse-over
parts of the image and be informed who is pictured. This functionality is also
used to add all posted images of someone to their profile.

\section{Wall}
A users wall stores all the posts of the user posted since the account was
created and the information about the user, this information is presented in
reverse chronological order, so that recent events are at the top of the page
and easily visible. Other users may view the users wall by clicking the name of
the user from anywhere in facebook. Other users may post on a friends wall along
with it's owner (see section on posts for more information); in this case, both
the poster and the owner of the wall can delete the post. Facebook also retains
the power to erase any content on its service.

Posts mentioning a user are automatically reposted to that users wall, this can
occur manually or when that person is tagged in an image.

\section{Chat System}
Facebook allows a user to chat with their friends, and will inform a user of
whether their friends are online or not (though this can be faked), and whether
the user you are chatting with has read the last message that you sent them. You
are also informed whether your friend is logged in on a mobile device or not.

Whole groups of users may chat together, in multi-user conversations. Facebook
also supports video calling and file transfer during chats. If a user does not
wish to be bothered by another using chatting with them, then they may 'mute'
that users conversion. Users spamming via chat may be reported to facebook.
Because multi-user conversations (and indeed long running one-to-one
conversations) can get rather large, facebook allows you to hide the history of
a conversation.

Facebook chat alerts the user to new messages in a conversation by playing a
sound.

\section{Architecture}
From a users point of view facebook is ostensibly organised as a single central
server; we are here concerned with the general architecture and not the
specific implementation of it, and so we will consider all of facebooks servers
to be a single server for the purposes of this section.

Users connect to facebook using a web browser, and proceed to download a client
written in javascript. User data is uploaded to facebook over HTTP as cleartext.
The data is stored on unencrypted on facebooks servers, and facebook maintains
a database of all data.

This allows clients to download only the data they need, as they can simply ask
for it. This in turn means that facebooks current architecture can, and does,
support a huge user-base, measured in the millions.


\section{Security}
In order to use facebook after registration a user must 'log in'. This places an
authentication cookie on the users computer which gives anyone in possession of
it the ability to act as that user.

If the user logs in from an IP associated with a region geographically far from
the last login, facebook will confirm that the user owns the account by asking
them to identify a friend in a photograph, or by other means.

Facebook chat turns the users computer into a server, whereby facebooks central
server sends messages to the client as it receives them, rather than the client
requesting new messages. This has been used in the past to identify facebook
users by correlating sent messages of specific size sent at a specific time.

Facebook has access to all its users data, and is able to erase, modify, and
fabricate it. Facebook is aware of everything which happens on facebook.
Censorship is a common occurrence on facebook.
