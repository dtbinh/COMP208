GPG is an implementation of the PGP\cite{rfc4880}, providing both public/private
key encryption and also a number of symmetric ciphers that can be used
separately.

It is common practice to use GPG to encrypt email, and several popular addons
for browsers exist to aid in this\cite{gpgaddon}. Unfortunately GPG itself is
difficult to use\cite{greenwaldAnnoying}, and a significant barrier to entry.

The encrypting of email with RSA\footnote{and a symmetric cipher} is a good
solution if one wants to keep the content of messages secure, and unmodified.
However it is out of scope for PGP to hide who is communicating, so while we
find the underlying cryptography sound, our scope is simply too different for
PGP to be of any use; with one exception.

Public key distribution is a significant challenge \footnote{Our system can't do
it, or it would be trivial to MitM users who don't check they received the
correct key via another channel}. PGP partially solves this problem by
introducing the concept of a 'web of trust'. In such a system one marks public
keys as trusted, presumably the keys of people you trust, and the people whom
you have marked as trusted can then sign the keys of other people whom they
trust. These keys may then be distributed, with the RSA signatures of everyone
who signed them, to everyone. If I download a key and see that it has been
verified by someone that I trust, then I can trust that key (albeit less than
the original key). This in combination with the small word hypothesis\footnote{
The phenomenon that people in the earth's population seem to be separated by at 
most 6 intermediaries.}\cite{sixdegrees} allows a large number of public keys to
become known to a user merely by adding one friends key, and having the client
automatically sign all keys it comes across from a trusted source.

We will take the 'web of trust' into consideration during design, however it
may present some significant security issues.
