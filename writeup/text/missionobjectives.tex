\section{projdesc-missionstatement}

\section{projdesc-missionobjectives}
The proposed project is a simple, privacy oriented social network, which demands
zero security knowledge on behalf of its users. In order to ensure security and
privacy in the face of nation state adversaries the system must be unable spy on
its users even if it wants to or server operators are ordered to.

We feel that obscuring the content of messages isn't enough, because suspician
may, and often does, fall upon people not for what they say, but to whom they
are speaking\todo{citation needed}. Our system will therefore not merely hide
the content of messages, but the recipient of messages too. Hiding the fact that
an IP address sent a message is out of scope, but hiding which user/keypair did
so is in scope.

The system will provide the following functionality:
\begin{itemize}
\item A user may add friends
\item A user may IM with fellow users
\item A user may IM anonymously with fellow users
\item A user may post messages to all their friends on their wall (think FB)
\item A user may request to post messages to all of a friends friends on a
friends page (think FB wall, permission is required because a user cannot know
who their friend (or anyone) is friends with)
\end{itemize}

The server operator will have access to the following information:
\begin{itemize}
\item Which IP uploaded which message (although they will be ignorant of its
content)
\item Which IPs are connecting to the server as clients
\item What times a specific IP connects \footnote {While this will aid in tying
an IP address to a person, it is deemed acceptable because it is not useful
information unless the persons private key is compromised.}
\end{itemize}

A third party logging all traffic between all clients and a server will have
access to the following information: \todo{Talk about TLS, end-to-end crypto}
\begin{itemize}
\item Who connects to the server, whether they upload or downloand information
\footnote{size correlation attacks could be used here if the message content is
known}
\end{itemize}

The benefits we feel this system provides over current solutions are the
following:
\begin{itemize}
\item Server operators can not know who talks with whom
\item Server operators can not know the content of messages
\item Server operators can not know which message is intended for which user
\item Server operators can not know who is friends with whom
\item Third parties sniffing the connections can not know anything the server
operator cannot know (this isn't unique, but is worth stating).
\end{itemize}

In order to ensure nobody can tell who is talking with whom we will base our
security model on the idea of shared mailboxes, as seen in practice at
alt.anonymous.messages
\footnote{https://groups.google.com/forum/\#!forum/alt.anonymous.messages}.
In this model one posts a message by encryping it using the public key of the
recipient, and posting it in a public location. In this model one reads a
message by downloading all messages from that location, and attempting to
decrypt them all using ones private key. Our protocol will build atop this
simple premise, and the the server will be a mere repository of messages, the
real work occuring wholly in the client.

\section{projdesc-projecttarget}
A project of this scope would normally have a target, outlined here in a more
formal manner. The project however, is ultimately not being done for an external
customer or supervisor, but for our own personal development. Whilst no formal
target is in mind, there are a number of possible recipents for this project.

Large multinational defence corporations such as IBM, Thales, or BAE might find
such a project useful, as it would allow for a secure communication tool between
employees in an office. It could also potentially be used outside a company
firewall to send messages securely between offices across much larger distances.
Corporations such as defence contractors often hold security in the highest
regard, and such a client would match their needs well.

A more likely recipent of this system however, is the internet itself, as we
have decided to release this project under an open source license. Should
another group decide to embark on a similar project, they will have access to
this project, to act as a baseline for their own work. More on this license
will be covered later in the document.
