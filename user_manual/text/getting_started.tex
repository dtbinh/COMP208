\section{Getting started}
Welcome to using Turtlenet!  Through the use of Turtlenet, you will experience
the ease of use and the practicality of communicating and socialising with your
friends, family, business associates or anyone else that you know through a
medium where your data is ensured to be protected.  This user manual has been
designed and written specifically to assist the users by providing detailed
description of all the various uses of the program.  Let's get started!

\section{System Requirements}
These are the minimum system requirements for Turtlenet:
\begin{itemize}
\item[Windows:]
\item Client: Windows Vista SP2 or greater; Server: Windows Server 2008 or 
      2012(64-bit)
\item Browser: Internet Explorer 7, Firefox, Chrome

\item[Mac OS X:]
\item Intel-based using Mac OS X 10.7.3 (Lion) or later
\item a 64-bit browser (Google Chrome may be a 32-bit browser, which could stop
                        Java from operating successfully.)
                        
\item[Linux:]
\item Any distribution capable of running a Java Virtual Machine (JVM) that is
      compatible with the official release of Java.
\item Our UI uses 'Google Web Toolkit' (GWT) which requires GTK2 (GIMP Tool Kit)
      so this may also be required for your system
      
\item[Source:]
\item Java Development Kit (JDK)
\item Java Runtime Environment (JRE)
\item Google Web Toolkit (GWT)
\item Apache Ant
\item GIMP Tool Kit 2 (GTK2)
\item GNU make / Capability of using makefiles
\item For Windows users, we suggest using 'Cygwin' for ease of use but there
      could be ways of building the source without it.
\end{itemize}

\section{The Turtlenet Interface}
Turtlenet comes with a simple interface that has the main menu, which has the
following sections:
\begin{itemize}
\item Login screen
\item My Wall
\item My Details
\item Messages
\item Friends
\item Logout
\end{itemize}

\section{Account Creation}
The user is expected to create a new account when using Turtlenet for the first
time.  In order to create an account, enter a user name and a password, as well
as repeating your password into the confirmation box.  Once the user has 
created an account, simply enter your password to login to Turtlenet.  From here
onwards, the user can then add further profile details should they wish to.  
How to do so will be explained under the 'Using the System' section.

\section{Transferring Keys}
talk about QR codes