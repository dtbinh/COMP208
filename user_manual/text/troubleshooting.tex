\section{Frequently Asked Questions}
This is the section which should hopefully answer most of the questions that
most users might have about the system.  Sending emails to one of the addresses 
in the contact section in the beginning of the user manual may help you get your
answer but it is best if you continue looking for an answer whilst you wait for
an official reply.

\subsection{What does Turtlenet do?}
Think of Turtlenet in a similar manner to any other social network commonly in
use.  It allows users to communicate with each other and allowing other people
to voice their opinions on what others have written.  At the moment it is text
based, meaning you can't attach images and video to it when you post or comment.
You can however send Universal Resource Links (URL) to each other, as they are
text based.  That is a convenient enough work around for the time being as it
means that no one is having to download an encrypted video but are never able to
view it as they do not have the key to unlock the data.  I think everyone will
appreciate not having current top 40 stored for a long time.

\subsection{How many accounts can I have on Turtlenet?}
As many as you like!  If you wish to have different personae within Turtlenet to
help filter friends from "It's kind of tricky I like them but they can be 
annoying" people then by all means - it's not on our head if they catch you 
putting them in the unmentionables list.

\subsection{I forgot my password.  Can someone reset it for me?}
The short answer is no.  Turtlenet was designed so that no one but the user had
access to personal data, protecting them from unwanted external influences.  As
a result, if you lose your password we are unable to recover anything in the
account.  The only thing you can do is simply to create another.  Feel safe in 
the knowledge that everything is encrypted on your old account so at least no 
one can access what was lost.

\subsection{Where is everything stored?}
On your computer, laptop or whatever else it is that uses the Turtlenet client.
Each client downloads all of the data and reads what it can, using keys you have
collected over time off of other users.  Keeping it local means that nothing is
stored on the server, so evil moderators cannot have their way with your data.
This also means that your data is stored on someone else's device but don't worry
 - just as you cannot read that user's information they cannot read yours.

\subsection{How big does this database get?}
As the only things being stored are text, not images or video, this means that
each message is only small and could only be about 4-8 gigabytes (GB) over one
year's use.  Bear in mind that the database is a replica of the entire history
of Turtlenet, along with everything every user has ever posted and commented on.
so much data for such a small size is pretty good.  In this time the project may
also have development applied to it, either from the original developers or the
community, so a cleaning function may be added to a future release.  We don't 
know.

\subsection{Why would someone want to build from source?}
For computers using Linux especially, the pre-built files may not work on their
system so in order to use Turtlenet, they would have to build the client
themselves.  Most Windows and Mac OS X users won't have to worry about this,
although if they want and know what to do, they are welcome to try!

\subsection{The Client does stuff I don't think it should do...}
You may have found a bug for us accidentally.  email to one of the addresses at
the beginning of the user manual and the developers will have a look at it.  As
the source is being released, maybe the community will have a look and suggest a
fix themselves.

\subsection{What do Server Moderators of Turtlenet do?}
Most of the time they watch text that they cannot understand go across the
screen.  This is to make sure that the server doesn't stop for some unknown 
reason.  They don't actually have any knowledge about what is being sent between
users so encode mildly evil messages and they will be none the wiser.  Unless
they have logged in as a user as well as moderating.  Then they still won't know
everything unless everyone adds the moderator's 'user account.'  A need-to-know
basis is what Turtlenet achieves.

\subsection{I want to mod Turtlenet.  Can I have the source?}
It's nice to know that others wish to take up the helm, pioneering a secure 
method of communication.  You can have the source, it is available to the public
to browse and modify.  One of the tools used in the initial development uses a
version of the GNU GPL, so be aware of what you can and can't do because of the
copyleft licence when you modify and distribute the project and it's source code.

\subsection{Why choose 'X' over the clearly superior 'Y'?}
As developers ourselves, we understand that other people have differing opinions.
That's the joy of releasing code.  Other people can pick up what we have done,
or use our ideals as a starting point for their own thing.  What this project
stood for is ease of use for the end user and security from any unwanted external
influences and this, we believe, is achieved.